
\subsubsection{MATLAB Scripting}

Due to the popularity of MATLAB in the neuroscience community, an
interface from MATLAB to TVB has been introduced that allows a MATLAB
script to design a TVB simulation, run it on TVB and retrieve the 
results. The MATLAB toolbox is provided separately from TVB, at
\url{https://github.com/the-virtual-brain/matlab-tvb}.

The implementation of this interface is a combination of an additional
CherryPy controller providing an HTTP/JSON API, running on the same 
server as the Web UI, and a set of MATLAB functions that send HTTP 
GET requests to the server. An implementation based on MEX functions 
invoking the Python library directly was considered, for reasons of 
performance, however, it was judged that such an implementation may be
difficult to stabilize and maintain, given that it would require binary
compatibility between MATLAB, Python and the C compiler. Two additional 
advantages of an HTTP API are that most computational environments have
the ability to connect and make HTTP requests, allowing other programs 
like Perl or Mathematica take advantage of TVB and the approach naturally
extends to work over the network, should TVB be running on another machine.

In the following, we give a short demonstration and 
describe implementation and rationale.

\lstinputlisting[caption={Matlab lsiting}, label={lst:uimatlab}]{matlab_lst.m}

Because the MATLAB functions need to know the address of the server,
so we take any of the URLs used by the Web UI (here, the one provided
when launching TVB):

%\begin{lstlisting}[caption={ML Listing},
%                   label={lst:ml1}]
%sv = vb_url('http://127.0.0.1:8080/user/')
%\end{lstlisting}

To run simulations without blocking MATLAB, a multiprocessing Pool
is used. We reset the pool and change the number of processes to 6

%\begin{lstlisting}[caption={ML Listing},label={lst:ml2}]
%vb_reset(sv, 6)
%\end{lstlisting}

Next, we can query the server for information on the classes available,
and also get help for each of the classes

%\begin{lstlisting}[caption={ML Listing},label={lst:ml2}]
%info = vb_dir(sv);
%\end{lstlisting}

\noindent where info is a cell array of structs, one per module (models,
monitors, etc.) and each struct has a field per class (models.JansenRit, 
models.Kuramoto, etc.). Each of these fields contains the details on 
the class, including all the parameters that can be set. 

To build a simulation, we start with an empty struct
and fill in the details for each part

%\begin{lstlisting}[caption={ML Listing},label={lst:ml3}]
%sim = [];
%
%sim.tf = 1e3 % simulation length milliseconds
%sim.model.class = vb.models.Generic2dOscillator;
%sim.model.a = -2.1;
%
%sim.connectivity.class = 'Connectivity';
%sim.connectivity.speed = 4.0;
%
%sim.coupling.class = 'Linear';
%sim.coupling.a = 0.002;
%
%sim.integrator.class = 'HeunDeterministic';
%sim.integrator.dt = 1e-2;
%\end{lstlisting}

Monitors are specified similarly but as a cell array there may be
several of them:

%\begin{lstlisting}[caption={ML Listing},
%                   label={lst:ml4}]
%sim.monitors{1}.class = 'TemporalAverage';
%
%sim.monitors{2}.class = 'Raw';
%sim.monitors{2}.period = 1.0; % ms
%\end{lstlisting}

\note[sk]{Raw monitor has no period, or rather the period can't be set as it is
fixed as the integration time step...}

Lastly, we submit the struct as a new simulation

%\begin{lstlisting}[caption={ML Listing},label={lst:ml5}]
%[id, data] = vb_new(sv, sim);
%\end{lstlisting}

\noindent Lastly, results are returned in a struct, here named data
where each field contians the output of a monitor and can be plotted
and analyzed as a regular MATLAB dataset:

%\begin{lstlisting}[caption={ML Listing},
%                   label={lst:ml6}]
%plot(data.mon_0_TemporalAverage.ts,...
%     squeeze(data.mon_0_TemporalAverage.ys)')
%\end{lstlisting}






