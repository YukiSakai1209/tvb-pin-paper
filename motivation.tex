
While whole-brain level simulators have been developed and published for
several years now, making the final step of connecting these simulations to
empirical results has remained a challenge due to several factors:

\begin{enumerate}
	\item Source code is typically not distributed, effectively closing
	the behavior, black box, etc. 

	\item The forward solutions required to obtain simulated M/EEG \& fMRI
	data are non trivial, requiring interaction with several pieces of software

	\item Published simulation methods for stochastic, delayed systems on
	surfaces are almost non existent

	\item Managing all of the different computational pieces is typically
	challenging for those who work with empirical data

\end{enumerate}

To address these concerns, a flexible architecture was developed to
allow easy integration of any computational tools along with a system
for describing typically types of data. A web based UI was developed
for users not comfortable with programming, as well as MATLAB toolbox
for interacting with the Python based framework, given that many
neuroscientists are already comfortable with the MATLAB workflow.
Lastly, a high performance, highly documented simulator along with
various forward solutions have been implemented and released under a
GPL licence to ensure universal access to high quality simulations, 
developed on the well-known Github, making it extremely easy for 
anyone to contribute.

\subsection{Why another simulator?}

This section actually motivates the integration story, otherwise 
nothing else is really necessary.

\subsection{Why Python?}

The core simulator actually began in MATLAB, however, as the needs 
expanded, the architecture, detailed in the next section, quickly 
outgrew the programming conventions typical of MATLAB packages.

Also, Python is free, and has an extremely rich ecosystem

\subsection{Collaboration}

In effect, we wish for a theoretician and clinician to be able to
collaborate; to enable such a possibility, we require a platform
to enable such opportunity. 

\subsection{Integration}

Large scale simulation implies flexible integration. We shall see
how this is enable by the architecture..

