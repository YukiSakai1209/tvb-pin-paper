
\TVB is divided into a framework and a scientific library, where the framework
handles primarily adapter infrastruct, the web-based user interface and
database connectivity while the scientific library includes datatypes, 
common analyses and the simulator. 

\subsection{Datatypes}

In scientific Python code, it is conventional to provide arguments
of an algorithm as a "bare" array or collection there of, and sanity
checks of arguments proceed on the basis of array geometry, for example.
In \TVB, we consider a \textit{datatype} to be a full, formal description of 
an entity involved in an algorithm that would be part of \TVB. 
For example, the \texttt{Connectivity} datatype, which may elsewhere
be represented by a simple $N$ by $N$ NumPy array, is written as a class
in which one of the attributes, \texttt{weights} is a explicitly typed 
\texttt{FloatArray}, and the declaration of this type is complemented by
explicit label, default values, and documentation strings. 

\note[mw]{Add a listing showing this}.

Because an explicit goal of \TVB was to provide a user interface to each of the
datatypes and algorithms contained within, it is necessary at some point to
provide metadata. A \texttt{traits} system was developed, similar to that of
IPython or EPD, was developed, allowing for a attributes of a datatype class to
be written out with full metadata. An extensive set of existing building 
blocks are already provided from numeric types and arrays to lists, tuples, 
string, and dictionaries.

\note[mw]{Show description of a data type}

 When methods of such a class are invoked,
they may use the traited attributes directly, accessing either a default value
or one given during the instantiation of the object. Additionally, this allows
the web-based UI to introspect a class for all of its attributes and their
descriptions, to provide help in the interface. The explicit typing also allows
such classes to be nearly automatically mapped to SqlAlchemy \& HDF5 tables,
the combination of which provides persistence for datatypes when using the web
UI.  Lastly, because such metadata is used to build the docstring of a class,
the IPython user also may obtain extensive descriptions of attributes and
function arguments in the usual way. 

\subsection{Adapters}

While datatypes provide a way of description what algorithms work with, 
sufficing for the typical user looking to write scripts against the
available libraries, the web-based UI requires algorithms to adhere to 
a generic interface, which is elsewhere referred to as the Adapter pattern.
Typically, this implies that a class is written that is able to describe
the collection to datatypes required and a single method to invoke the
algorithm.

\note[mw]{Show the adapter for FastICA}

Note that the adapters and datatypes are intended to provide full 
power and flexibility of the framework; when the simulator is invoked from
the web-based UI, it is done so through an \texttt{SimulatorAdapter} which,
despite being relatively complex, is datatypes all the way down.

\subsection{Examples}

\note[mw]{Show code examples, and explain the mechanics}

\subsubsection{Adapting sklearn's FastICA}

\subsubsection{Interfacing with the Brain Connectivity Toolbox}

\subsubsection{The Surface datatype}



