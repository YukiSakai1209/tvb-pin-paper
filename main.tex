%!TEX TS-program = pdflatex                                                    %
%!TEX encoding = UTF8                                                          %
%!TEX spellcheck = en-US                                                       %
%------------------------------------------------------------------------------%
% to compile use "latexmk --pdf main.tex"                                      %
%------------------------------------------------------------------------------%
% to count words 
% "pdftotext main_nofigs_nocaptions.pdf - | egrep -e '\w\w\w+' | iconv -f ISO-8859-15 -t UTF-8 | wc -w"
% -----------------------------------------------------------------------------%

\documentclass{bioinfo}
\usepackage{url}
\usepackage[english]{babel}
\usepackage[utf8]{inputenc}
\usepackage[T1]{fontenc}
%\usepackage[pdftex]{graphicx} 
%\usepackage{graphics}
%\usepackage{hyperref}
\usepackage{float}
\floatplacement{figure}{H}
\usepackage{booktabs}     % nice tables
\usepackage{tabularx}     % even nicer tabular environments 
\usepackage{amsmath}
\usepackage{amsfonts}
\usepackage{amssymb}
%\usepackage{multicol}
\usepackage{listings}
\usepackage{tikz,times}
\usepackage{courier}
\usetikzlibrary{shapes,arrows}
\usetikzlibrary{arrows,positioning}
\usepackage{xcolor}
\usepackage[font=bf]{subfig}
%\usepackage{sectsty}
%\sectionfont{\normalsize\bfseries}
%\usepackage[labelfont=bf]{caption}
%\usepackage{endfloat} %place figures at end of document
%------------------------------------------------------------------------------%
%\captionsetup{
%%format = hang,                % caption format
%labelformat = simple,          % caption label : name and number
%labelsep = period,             % separation between label and text
%textformat = simple,           % caption text as it is
%justification = justified,     % caption text justified
%singlelinecheck = true,        % for single line caption text is centered
%font = {up,singlespacing},     % defines caption (label & text) font
%labelfont = {bf,footnotesize}, % NOTE: tiny size is not working
%textfont = footnotesize,
%%width = \textwidth,           % define width of the caption text
%skip = 1ex,                    % skip the space between float and caption
%listformat = simple,           % in the list of floats, label + caption
%}

%------------------------------------------------------------------------------%
%\hypersetup{
%    bookmarks=true,         % show bookmarks bar?
%    unicode=false,          % non-Latin characters in Acrobat’s bookmarks
%    pdftoolbar=true,        % show Acrobat’s toolbar?
%    pdfmenubar=true,        % show Acrobat’s menu?
%    pdffitwindow=false,     % window fit to page when opened
%    pdfstartview={FitH},    % fits the width of the page to the window
%    pdftitle={TheVirtualBain},    % title
%    pdfauthor={PSL},        % author
%    pdfsubject={ProposedArticle},   % subject of the document
%    pdfcreator={paupau},    % creator of the document
%    pdfnewwindow=true,      % links in new window
%    colorlinks=true,       % false: boxed links; true: colored links
%    linkcolor=red,          % color of internal links (change box color with linkbordercolor)
%    citecolor=blue,        % color of links to bibliography
%    filecolor=magenta,      % color of file links
%    urlcolor=blue           % color of external links
%}
%-----------------------------------------------------------------------
%\usepackage{subcaption}

%%%%%%%%%%%%%%%%%%%%%%%%%%%%%%%%%%%%%%%%%%%%%%%%%%%%%%%%%%%%%%%%%%%%%%%%%%%%%%%%
%%                             New and renew commands                         %%
%%%%%%%%%%%%%%%%%%%%%%%%%%%%%%%%%%%%%%%%%%%%%%%%%%%%%%%%%%%%%%%%%%%%%%%%%%%%%%%%
  
\renewcommand{\lstlistingname}{Code}
\renewcommand{\thesubfigure}{\Alph{subfigure}}
\newcommand{\inputTikZ}[2]{%
\scalebox{#1}{\input{#2}}}
\newcommand*{\h}{\hspace{5pt}}   % for indentation
\newcommand*{\hh}{\h\h}          % double indentation
\newcommand{\TVB}{\textit{TheVirtualBrain }}
\newcommand*{\tvbmodule}[1]{{\textsc{#1}}}          % scientific modules in "simulator"
\newcommand*{\tvbdatatype}[1]{\textbf{\emph{#1}}}   % datatypes in "datatypes"
\newcommand*{\tvbclass}[1]{{\ttfamily\emph{#1}}}    % classes either in simulator mods or datatypes
\newcommand*{\tvbmethod}[1]{{\textsf{#1}}}          % methods
\newcommand*{\tvbattribute}[1]{{\ttfamily{#1}}}     % attributes
\newcommand*{\tvbtrait}[1]{{\ttfamily{#1}}}         % traited types



%%%%%%%%%%%%%%%%%%%%%%%%%%%%%%%%%%%%%%%%%%%%%%%%%%%%%%%%%%%%%%%%%%%%%%%%%%%%%%%%
%%                            Colors and graphics                             %%
%%%%%%%%%%%%%%%%%%%%%%%%%%%%%%%%%%%%%%%%%%%%%%%%%%%%%%%%%%%%%%%%%%%%%%%%%%%%%%%%
\definecolor{palegreen}{HTML}{DAFFDA}
\definecolor{lightgray}{rgb}{0.15,0.15,0.15}
\definecolor{orange}{HTML}{FF7300}
\DeclareGraphicsExtensions{.jpg,.pdf,.png,.tiff}%,.mps,.bmp
\graphicspath{{figures/}}
 
%##--------------------------------------------------------------------------##%
%##                               START HERE                                 ##%
%##--------------------------------------------------------------------------##%
\copyrightyear{}
\pubyear{}

\begin{document}
\lstset{language=Python, 
        caption=b, 
        breaklines=false, 
        basicstyle=\bf\tiny\ttfamily, 
        stringstyle=\color{magenta}
        } 
\firstpage{1}

%%  Authorship and Title
\title[TVB]{Integrating neuroinformatics tools in TheVirtualBrain}
\author[Sanz Leon {et~al}]{
        M. Marmaduke Woodman\,$^{1,*}$,  
        Paula Sanz Leon\,$^{1}$, 
        Jochen Mersmann\,$^{2}$,
        Lia Domide\,$^{3}$, 
        Anthony R. McIntosh \,$^{4}$ and  
        Viktor Jirsa\,$^{1}$\footnote{to whom correspondence should be addressed: mw@eml.cc,
        viktor.jirsa@univ-amu.fr}}

\address{$^{1}$ Institut de Neurosciences des Syst{\`e}mes, 27, Bd. Jean Moulin, 13005, Marseille, France.\\
         $^{3}$ Codemart, 13, Petofi Sandor, 400610, Cluj-Napoca, Romania.\\
         $^{2}$ CodeBox GmbH, Hugo Eckener Str. 7, 70184 Stuttgart, Germany.\\
         $^{4}$ Rotman Research Institute at Baycrest, Toronto, M6A 2E1, Ontario, Canada\\
        }

\history{}

\editor{}

\maketitle

%##--------------------------------------------------------------------------##%
%##                               ABSTRACT                                   ##%
%##--------------------------------------------------------------------------##%


\begin{abstract}
\section{}

TheVirtualBrain (TVB) is a neuroinformatics Python package representing the
convergence of lines of work in clinical, systems, theoretical neuroscience in
the integration, analysis, visualization and modeling of neural dynamics of
the human brain as well as the imaging modalities through which these dynamics
are measured. Specifically, TVB is composed of a flexible simulator for both 
neural dynamics and modalities such as MEG and fMRI, common analysis 
techniques such as wavelet decomposition and multiscale sample entropy, 
interactive visualizers for replaying cortical timeseries on the 3D surface 
or editing large-scale connectivity matrices, and an (optional) user interface
accessible through modern web browsers. Tying together these
pieces with persistent data storage, based on a combination of SQL \& HDF5,
is a rich, open-ended system of datatypes modeling (systems level) neuroscientific data and the relations
among them. This data modeling system in parallel with the so-called 
adapter pattern architecture permit the integration of TVB with any other
computational system, including MATLAB for which support is already available.
 Notably, TVB provides infrastructure for multiple projects and 
multiple users, possibly participating under multiple roles: a clinician
may import diffusion spectrum imaging data, launch a tractography algorithm, 
and identify potential lesion points, and then share this data with a computational 
expert who would then enter to contribute simulation parameter sweeps and analyses, 
to test which lesion point is most probably given certain empirical imaging data, 
et cetera; this is one of many multi-user use cases supported by TVB.
TVB also drives research forward on many levels: the simulator
itself represents the systematization of several recent ad-hoc simulations
in the modeling literature on human rest state. 
In these ways, TVB serves as an integrating platform for disparate expertises
in the high level analysis and modeling of the human brain.
This paper will begin with a brief outline of the history and motivation for TVB
as a unified project \textit{per se}. We proceed to describe the framework and simulator, 
giving usage examples in the web UI and in plain Python scripting.
Finally, we compare TVB with the nearest neighbors in brain modeling, simulation
performance, recent advances thereupon with native code compilation and GPUs, and the 
role of Python and its rich scientific ecosystem in TVB. 

  
\section{Keywords:} connectivity, connectome, neural mass, neural field, 
time delays, full-brain network model, Python, virtual brain, large-scale 
simulation,  web platform, GPUs
\end{abstract}

\end{document}
