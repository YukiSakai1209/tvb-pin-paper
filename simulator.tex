The \TVB simulator resembles popular neural network simulators in 
many fundamental ways, both mathematically and in terms of informatics 
structures, however we have found it necessary to introduce auxiliary
concepts particularly useful in the modeling of large scale brain 
networks.

\subsection{Node dynamics}

	In our models, nodes are not abstract neurons nor necessarily 
	small groups thereof, but rather large populations of neurons, considered
	by, for example, the work of Wilson and Cowan. 

	\note[mw]{quick list of models \& their relevance}

\subsection{Network structure}
	
	The nodes are embedded in two scale network structure. The first of which
	is derived from empirical measurements of the myelinated corticocortical
	connectivity, which we shall refer to as the inhomogeneous, large scale
	structure. An implication of this structure is the inherent delays in 
	communication due to finite conduction velocity.

	The second form of network structure is prescribed in the case of a 
	cortical surface by the combination of said surface and a connectivity
	kernel. Together, these generate a homogeneous, local connectivity. 
	For the current work, we consider this coupling to be instantaneous.

	\note[mw]{Need to explain regional \& surface simulations}

\subsection{Integration of stochastic delay differential equations}

	Rather unlike other simulation paradigms, both noise and delays are
	common features of simulations in \TVB. As such, the simulator is equipped
	with Heun and Euler methods for stochastic integration and the pairwise
	inter-regional delays are treated in an efficient fashion. 

	\note[mw]{the general equation here}

	\note[mw]{Describe handling of delays?}

\subsection{Forward solutions}

	One of the primary goals of the \TVB simulator is to allow for the
	simulation of empirical whole-brain data, such as EEG or fMRI.
	\TVB implements several forward solutions as so-called monitors, 
	including MEG, sEEG, EEG and fMRI. 

	Crucially, given the amount of data \TVB may produce, especially for
	simulations on a cortical surface, each of the monitors is "on-line"
	in the sense that is runs in constant space.

\subsection{Native code generation for C \& GPU}

	Several of the core components (integrators, mass models, coupling
	functions) have targeted towards a C source code backend, which has
	allowed for the compilation of simulations to native code loaded 
	either as a shared library accessed via the \texttt{ctypes} modules
	or as CUDA kernels accessed via the \texttt{pycuda} module \cite{pycuda}.
	While such an approach may provide speed ups, they depend on the
	presence of a C compiler and, in the case of GPU, the CUDA toolkit and
	a compatible graphics card. 


\subsection{Other simulators compared to \TVB}

	Brian should be a particular focus in this section, as it may
	be one of the closest. 

